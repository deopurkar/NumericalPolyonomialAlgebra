\documentclass[11pt]{article}
\usepackage{
  amsmath,
  amssymb,
  tikz,
  microtype,
  amsthm,
  thmtools,
  stmaryrd,
  hyperref,
  algorithm2e
}


\title{Special topics course: Numerical Polynomial Algebra}
\author{Anand Deopurkar and Markus Hegland}
\date{ANU winter semester 2018}

\renewcommand{\algorithmautorefname}{Algorithm}
\newcommand{\C}{\mathbb C}
\newcommand{\Z}{\mathbb Z}
\newcommand{\R}{\mathbb R}
\DeclareMathOperator{\LT}{LT}
\DeclareMathOperator{\lcm}{lcm}
\DeclareMathOperator{\wt}{wt}

\declaretheorem{theorem}
\declaretheorem[sibling=theorem, style=theorem]{proposition}
\declaretheorem[sibling=theorem, style=definition]{definition}
\declaretheorem[sibling=theorem, style=definition]{example}
\declaretheorem[sibling=theorem, style=definition]{problem}
\declaretheorem[sibling=theorem, style=remark]{remark}

\begin{document}

\maketitle

\part*{Algebraic introduction}
This is a quick introduction to the commutative algebra relevant to systems of polynomial equations.
Our main references are Stetter \cite{stetter04} and Cox, Little, O'Shea \cite{cox15}.

\tableofcontents

\section{Polynomials and ideals}

Let $k$ be a field.
Denote by $k[x_1,\dots,x_s]$ the set of polynomials in variables $x_1, \dots, x_s$ with coefficients in $k$. 
Stetter denotes $k[x_1,\dots,x_s]$ by $P^s_k$.
For numerical analysis, we will take the base-field $k = \R$, but a lot of the algebraic geometry works best with $k = \C$.
The case of $k = \Z/p\Z$ is used frequently for exact symbolic computation.
We will try to be base-field agnostic; but for some results, the base-field will play an important role. 

Let $f_1, \dots, f_k$ be elements of $k[x_1,\dots, x_s]$.
We are interested in solving the system of polynomial equations
\begin{align}\label{eqn:system}
  \begin{split}
  f_1(x_1, \dots, x_s) &= 0,\\
  &\vdots\\
  f_n(x_1, \dots, x_s) &= 0.
\end{split}
\end{align}
Our experience with systems of linear equation suggests that we should look at not only the $n$ equations listed above, but also their linear combinations.
This idea suggests the following.
\begin{definition}[Ideal]
  A subset $I$ of $k[x_1, \dots, x_s]$ is called an \emph{ideal} if the following two conditions are satisfied.
  \begin{enumerate}
  \item If $f, g \in I$, then $f+g \in I$.
  \item If $f \in I$ and $c \in k[x_1, \dots, x_s]$, then $cf \in I$.
  \end{enumerate}
\end{definition}
The definition of an ideal should be reminiscent of the definition of a subspace of a vector space.
The only difference is that the multipliers in the second condition are not necessarily scalars (elements of $k$) but polynomials.
By putting the two conditions together, we see that an ideal is a subset of $k[x_1, \dots, x_s]$ that is closed under polynomial linear combinations.
That is, if $f_1, \dots, f_n \in I$ and $c_1, \dots, c_n \in k[x_1, \dots, x_s]$, then $c_1f_1 + \dots + c_nf_n \in I$.

\begin{example}[The unit ideal]
  The set $I = k[x_1,\dots,x_s]$ is obviously an ideal; it is called the unit ideal.
  Observe that it is the only ideal that contains $1$.
\end{example}

\begin{example}[Ideal generated by a set of polynomials]
  If $f_1, \dots, f_n$ are elements of $k[x_1, \dots, x_s]$, then the set consisting of all polynomial linear combinations of $f_1, \dots, f_n$, namely
  \[ I = \{c_1f_1 + \dots + c_nf_n \mid c_1, \dots, c_n \in k[x_1,\dots,x_s]\}\]
  is an ideal.
  It is the smallest ideal containing $f_1, \dots, f_n$.
  We say that $I$ is \emph{generated by} $f_1, \dots, f_n$, and write
  \[ I = \langle  f_1, \dots, f_n \rangle.\]
\end{example}

A fundamental theorem in commutative algebra is that every ideal has a finite generating set.
\begin{theorem}[Hilbert basis theorem]
  Let $I \subset k[x_1, \dots, x_s]$ be an ideal.
  Then there exists polynomials $f_1, \dots, f_n$ such that $I = \langle  f_1, \dots, f_n \rangle$.
\end{theorem}
There are multiple generating sets for a given ideal, just as there are multiple spanning sets for a subspace of a vector space.
Depending on what we want to do, some generating sets are more useful than others.
A common theme in the course will be to learn how to construct and use good generating sets.

\subsection{Zero loci of ideals}
Let $I \subset k[x_1, \dots, x_s]$ be an ideal.
We can associate to $I$ a subset $V(I)$ of $k^s$ called the \emph{zero locus} or the \emph{vanishing locus} of $I$.
It is defined as follows:
\[
  V(I) = \{x \in k^s \mid f(x) = 0 \text{ for all $f \in I$}\}.
\]
Observe that if $I = \langle  f_1, \dots, f_n \rangle$, then $V(I)$ is simply the set of solutions of the system of equations \eqref{eqn:system}.
Thus, solving \eqref{eqn:system} is equivalent to finding $V(I)$.

A subset of $k^s$ of the form $V(I)$ for some ideal $I$ is called an \emph{algebraic set}.
Some authors also call an algebraic set an \emph{algebraic variety}, but some other authors reserve the term \emph{variety} only for special types of algebraic sets (namely, the ``irreducible'' ones).

We can turn the construction $I \leadsto V(I)$ around.
Starting with a subset $Z \subset k^s$, we can construct an ideal $I(Z) \subset k[x_1, \dots, x_s]$ as follows:
\[ I(Z) = \{f \in k[x_1,\dots,x_s] \mid f(x) = 0 \text{ for all $x \in Z$}\}.\]
It is easy to check that $I(Z)$ is an ideal.
Furthermore, it is easy to verify that $I(Z)$ is a \emph{radical} ideal.
That is, it satisfies the property that if $f^n \in I(Z)$ for some $n \geq 1$, then $f \in I(Z)$.
(This property is not shared by all ideals.
For example, the ideal $\langle  x^2 \rangle \subset k[x]$ is not radical---it contains $x^2$, but not $x$.)

\begin{example}[Ideal of $3$ points in $k^2$]
  \label{ex:3pts}
  Let $Z = \{(0,0), (0,1), (1,0)\}$.
  Then we have
  \[ I(Z) = \langle x(x-1), xy, y(y-1)) \rangle.\]
\end{example}
\begin{example}[Ideal of the ``twisted cubic'']
    \label{ex:twistedcubic}
Let $Z = \{(t, t^2, t^3) \mid t \in \C\} \subset \C^3$.
  Then
  \[ I(Z) = \langle  x^2-y, y^2-xz, x^3-z \rangle.\]
\end{example}
\begin{example}[Ideal of $4$ points in $k^2$]
  \label{ex:4pts}
  Let $Z = \{(0,0), (1,0), (0,1), (1,1)\}$.
  Then
  \[ I(Z) = \langle  x(x-1), y(y-1) \rangle.\]
\end{example}
\begin{remark}
  The ideal in \autoref{ex:4pts} is different in a crucial way from the ideals in \autoref{ex:3pts} and \autoref{ex:twistedcubic}.
  It is a \emph{complete intersection} ideal.
  A complete intersection ideal is an ideal that is generated by as many polynomials as the codimension of its vanishing set.
  In \autoref{ex:4pts}, the codimension of the vanishing set is 2, and indeed, the ideal is generated by 2 polynomials.
  One can prove that if the base-field is algebraically closed, then an ideal needs at least as many generators as the codimension of its vanishing locus.
  A complete intersection ideal is one where the bound is achieved.

  A complete intersection ideal is stable under perturbation of its generators.
  We will not make this precise, but illustrate it in the examples above.
  If we perturb one (or both) of the generators of the ideal in \autoref{ex:4pts}, then the perturbed ideal remains the ideal of $4$ points.
  In contrast, if we perturb the generators of the ideals in \autoref{ex:3pts} or \autoref{ex:twistedcubic}, then the ideal changes dramatically in shape.
  In \autoref{ex:3pts}, a perturbation of the generators will make the ideal become the unit ideal.
  In \autoref{ex:twistedcubic}, a perturbation of the generators will make the ideal become the ideal of a finite set of points.
  You are encouraged to verify these statements about perturbations either by hand or using a computer algebra system.
\end{remark}

\subsection{The Nullstellensatz}
The following important theorem links the constuctions $I \mapsto V(I)$ and $Z \mapsto I(Z)$.
\begin{theorem}[Nullstellensatz]
  Suppose $k$ is an algebraically closed field (for example, $k = \C$).
  Then we have a one-to-one correspondence
  \[
    \{\text{Radical ideals of $k[x_1,\dots,x_s]$}\}
    \leftrightarrow
    \{\text{Algebraic subsets of $k^s$}\},
  \]
  where the $\rightarrow$  direction is given by $I \mapsto V(I)$ and the $\leftarrow$ direction is given by $I(Z) \mapsfrom Z$.
\end{theorem}

\begin{example}[Failure of the Nullstellensatz for non-closed fields]
  In the Nullstellensatz, it is essential for the base-field to be algebraically closed.
  To see what can go wrong otherwise, take $k = \R$ and $I = \langle  x^2 + 1 \rangle \subset k[x]$.
  Then $I$ is a radical ideal, and we have $V(I) = \emptyset$.
  But we also have $V(\langle  1 \rangle) = \emptyset$.
  So the correspondence is not one-to-one.

  This example shows that when $k$ is ``too small'' (not algebraically closed), then the geometric (right) side may fail to faithfully capture the algebraic (left) side.
  But this happens only because the points on the right hand side that \emph{should} be there are not there because the base-field is too small.
\end{example}

A consequence of the Nullstellensatz is the following dichotomy for systems of polynomial equations.
\begin{theorem}\label{thm:null}
  Consider the system of polynomial equations $f_1 = 0, \dots, f_n = 0$ over an algebraically closed field $k$.
  Then either there exists a solution or there exist polynomials $c_1, \dots, c_n$ such that
  \[ 1 = c_1 f_1 + \dots + c_nf_n.\]
  Equivalently, given an ideal $I$, either $V(I)$ is non-empty, or $I = \langle  1 \rangle$.
\end{theorem}
\begin{proof}
  Suppose $V(I)$ is empty.
  We must prove that $1 \in I$.
  If $I$ is radical, then we have
  \[ V(I) = V(\langle 1 \rangle) = \emptyset.\]
  So by the Nullstellensatz, we have $I = \langle  1 \rangle$, and we are done.

  If $I$ is not radical, then we replace it by its radical $\sqrt I$, defined by
  \[ \sqrt I = \{f \mid f^n \in I \text{ for some $n \geq 1$}\}.\]
  It is not too hard to check (do it!) that $\sqrt I$ is an ideal, is radical, and $V(I) = V(\sqrt I)$.
  From the argument above, we get that $\sqrt I = \langle  1 \rangle$, that is $1 \in \sqrt I$.
  But by the definition of $\sqrt I$, this means that $1^n \in I$ for some $n \geq 1$, so $I = \langle  1 \rangle$.
\end{proof}

\begin{remark}
  Over the real numbers, although the Nullstellensatz fails, an analogous theorem called the Positivestellensatz holds.
  Although it is tempting, we will not discuss this further.
  This topic belongs to an interesting field of mathematics that lies on the interface of algebraic geometry, combinatorics, applied mathematics, and logic, called \emph{real algebraic geometry}.
\end{remark}

\section{Ideal membership and the division algorithm}
Let us return to the system of polynomial equations
\begin{align}\label{eqn:system1}
  \begin{split}
  f_1(x_1, \dots, x_s) &= 0,\\
  &\vdots\\
  f_n(x_1, \dots, x_s) &= 0.
\end{split}
\end{align}
As a first step towards solving this system, we would like to determine whether the system has any solutions at all.
By \autoref{thm:null}, this is equivalent to determining whether $1$ lies in the ideal $\langle  f_1, \dots, f_n \rangle$.
In this section, we will consider a general version of this problem.
\begin{problem}[Ideal membership]
  Given polynomials $f_1, \dots, f_n$ and a polynomial $g$, determine whether $g$ lies in the ideal $\langle  f_1, \dots, f_n \rangle$.
\end{problem}

\subsection{The one-variable case}
Suppose $s = 1$ and $n = 1$.
So, we are given a (non-zero) polynomial $f(x)$ and another polynomial $g(x)$.
We want to determine whether $g(x)$ is lies in the ideal $\langle  f(x) \rangle$, that is, whether $g(x)$ is a multiple of $f(x)$.
Here we know what to do: simply run the division algorithm (\autoref{alg:divalg1}).

\begin{algorithm}
  \While{
    $\deg (g) \geq \deg(f)$
  }
  {
    Replace $g$ by $g - \frac{\LT(f)}{\LT(g)} f$, where $\LT$ denotes the leading (highest degree) term.
  }
  Return $g$ as the remainder.
  \bigskip

  \caption{The division algorithm for $g(x) \pmod {f(x)}$}
  \label{alg:divalg1}
\end{algorithm}

In the one-variable case, we have the following key fact.
\begin{proposition}
  The polynomial $g(x)$ lies in $\langle  f(x) \rangle$ if and only if the remainder obtained in the division algorithm for $g(x) \pmod {f(x)}$ is zero.
\end{proposition}

\subsection{Term orderings}
To extend the division algorithm to handle more than one variable and ideals generated by more than one polynomial, we must first extend the notion of the leading term.
It turns out that there are multiple ways to do this.
It gives rise to the idea of a term ordering.

Let $T^s$ be the set of monomials in the variables $\{x_1, \dots, x_s\}$.
We use multi-indices to denote the elements in $T^s$.
For a multi-index $i = (i_1, \dots, i_s)$, the notation $x^i$ means the monomial $x_1^{i_1} \cdots x_s^{i_s}$.
\begin{definition}[Term ordering]
  A \emph{term ordering} is a total ordering $\prec$ on the elements of $T^s$ satisfying the following two properties:
  \begin{enumerate}
  \item if $x^i \prec x^j$ then $x^i \cdot x^k \prec x^j \cdot x^k$ for all multi-indices $k$.
  \item $1 \prec x^i$ for all multi-indices $i$.
  \end{enumerate}
\end{definition}

For $s = 1$, there is only one possible term ordering, namely \[1 \prec x \prec x^2 \prec \cdots.\]
For $s > 1$, however, there are many possibilities.
\begin{example}[Lexicographical ordering $\prec_{\rm lex}$]
  Fix an ordering on the variables, say $x_s \prec x_{s-1} \prec \cdots \prec x_1$.
  Extend this ordering to monomials using the lexicographical rule.
  That is, set $x^i \prec_{\rm lex} x^j$ if the first non-zero entry of $j-i$ is positive.
\end{example}

\begin{example}[Graded lexicographical ordering $\prec_{\rm grlex}$]
  Fix an ordering on the variables, say $x_s \prec x_{s-1} \prec \cdots \prec x_1$.
  Extend this ordering to monomials by first comparing the total degree and then using the lexicographical rule.
  That is, set $x^i \prec_{\rm grlex} x^j$ if $\deg(x^i) < \deg(x^j)$ or if $\deg(x^i) = \deg(x^j)$ and $x^i \prec_{\rm lex} x^j$.
\end{example}

\begin{example}[Weight ordering]
  Assign a weight $\alpha_i \in \R_{> 0}$ to each variable $x_i$.
  Extend the weight assignment to monomials by taking a weighted sum
  \[ \wt(x^i) = \sum_{\ell = 1}^s i_\ell \alpha_\ell.\]
  Use this to define a term ordering by setting $x^i \prec_\alpha x^j$ if $\wt(x^i) < \wt(x^j)$.

  We have to take care that there are no ties in the weights of the monomials (or devise a tie-breaker).
  For generic (almost all) choices of weights, there will be no ties.
\end{example}

Fix a term ordering $\prec$.
Using $\prec$, we can define the leading term of a polynomial.
For $f \in k[x_1,\dots,x_s]$, let $\LT(f)$, or more precisely $\LT_\prec(f)$, be the term of $f$ with the highest monomial (according to $\prec$).
For example, we have
\begin{align*}
  \LT_{\prec_{\rm lex}} (5x^3y^3-3x^4)  &= -3x^4 \\
  \LT_{\prec_{\rm grlex}} (5x^3y^3-3x^4)  &= 5x^3y^3. 
\end{align*}
We can now generalize the division algorithm as follows.

\begin{algorithm}
  \While{
    there exists an $i$ such that $\LT(f_i)$ divides $\LT(g)$.
  }
  {
    Replace $g$ by $g - \frac{\LT(f_i)}{\LT(g)} f_i$.
  }
  Return $g$ as the remainder.
  \bigskip

  \caption{The division algorithm for $g \pmod {f_1, \dots, f_n}$}
\end{algorithm}

It is easy to see that if the division algorithm for $g \pmod {f_1, \dots, f_n}$ returns $0$ as the remainder, then $g$ is in the ideal $\langle  f_1, \dots, f_n \rangle$.
However, the converse is not true, as the next examples show.
This severely diminishes the utility of the division algorithm.

The division algorithm suffers from another drawback.
Note that there may be choices involved in the running of the algorithm.
Namely, if there are multiple $i$'s such that $\LT(f_i)$ divides $\LT(g)$, then we may choose any one and proceed.
The final remainder obtained may change depending on the choices made in the algorithm.

Nevertheless, we will see in the next section, that we can overcome both o these drawbacks by making a judicious choice of the generating set for our ideal.

\begin{example}[Failure of the division algorithm]
  \label{eq:fail1}
  Let us work with $\prec_{\rm grlex}$.
  Take $f_1  = x^3-2xy$ and $f_2 = x^2y-2y^2+x$; and $g = x^2$.
  Then the division algorithm stops without any computation, and returns $x^2$ as the remainder.
  However, the following computation shows that $g$ is actually in the ideal $\langle  f_1, f_2 \rangle$:
  \[ g = x \cdot f_2 - f_1.\]
\end{example}

\begin{example}[Failure of the division algorithm]
  We can work in any term ordering.
  Take $f_1 = xy-1$, $f_2 = x^3-y^3$, and $f_3 = x^2+y^2$.
  Check that $1 \in \langle  f_1, f_2, f_3 \rangle$ (either directly or using that $V(\langle  f_1, f_2, f_3 \rangle) = \emptyset$).
  However, the division algorithm stops without any computation, and returns $1$ as the remainder.
\end{example}

\subsection{Gr\"obner bases}
Let $I \subset k[x_1,\dots, x_s]$ be an ideal and $\prec$ a term ordering.
The \emph{initial ideal} of $I$ with respect to $\prec$ is the ideal generated by the leading terms of the polynomials in $I$:
\[
  \LT(I) = \langle  \LT(f) \mid f \in I \rangle.
\]
Note that if $I = \langle  f_1, \dots, f_n \rangle$, then
\[ \langle \LT(f_1), \dots, \LT(f_n) \rangle \subset \LT(I),\]
but this may be a strict inclusion.
\begin{example}
  Consider the ideal from \autoref{eq:fail1},
  \[ I = \langle  x^3-2xy, x^2y-2y^2+x \rangle.\]
  With respect to $\prec = \prec_{\rm grlex}$, we have
  \[ \LT(x^3-2xy) = x^3 \text{ and } \LT(x^2y-2y^2+x) = x^2y.\]
  As a result, the inclusion
  \[ \langle \LT(x^3-2xy) , \LT(x^2y-2y^2+x) \rangle \subset \LT(I)\]
  is strict.
  The polynomial $x^2$ is contained in the ideal on the right (because $x^2 \in I$), but not contained in the ideal on the left.  
\end{example}
\begin{definition}[Gr\"obner basis]
  Let $f_1, \dots, f_n$ be elements of $I$.
  We say that $\{f_1, \dots, f_n\}$ is a Gr\"obner basis of $I$ (with respect to $\prec$) if
  \[ \langle  LT(f_1), \dots, \LT(f_n) \rangle = \LT(I).\]
\end{definition}
\begin{remark}
  The term ``basis'' is used more loosely in polynomial algebra than in linear algebra.
  A set of generators of an ideal is often called a basis.
  Unlike a basis in linear algebra, a basis of an ideal is never ``linearly independent'', unless it consists of a single polynomial.
  Indeed, if $f_1, f_2$ are two elements of the basis, then they always satisfy the (polynomial) linear relation
  \[ f_2 \cdot f_1 - f_1 \cdot f_2 = 0.\]
\end{remark}
Using the Hilbert basis theorem, we can show that every ideal has a (finite) Gr\"obner basis.
Even better, we will see an algorithm to construct a Gr\"obner basis and test whether a given set of generators form a Gr\"obner basis.
But let us postpone this, and see why a Gr\"obner basis is useful.

First, running the division algorithm with a Gr\"obner basis always works.
Throughout, we fix a term ordering $\prec$.
\begin{proposition}\label{prop:divgrob}
  Let $f_1, \dots, f_n$ be a Gr\"obner basis for $I$.
  Let $g$ be the remainder given by the division algorithm for $f \pmod {f_1, \dots, f_n}$.
  Then $g = 0$ if and only if $f \in I$.
  Furthermore, $g$ does not depend on the choices made during the running of the algorithm.
\end{proposition}
\begin{proof}
  If $g = 0$, then it is clear that $f \in I$.
  The content is in the converse.

  Suppose $g \neq 0$.
  Then we need to show that $f \not \in I$.
  Note that we have $f-g \in I$.
  So, to prove that $f \not \in I$, it suffices to prove that $g \not \in I$.
  Since $g$ is the remainder, by the design of the algorithm, $\LT(f_i)$ does not divide $\LT(g)$ for any $i = 1,\dots, n$.
  Since $\LT(f_i)$ and $\LT(g)$ are monomials, it is easy to check that this implies
  \[ \LT(g) \not \in \langle  \LT(f_1), \dots, \LT(f_n) \rangle.\]
  Since $f_1, \dots, f_n$ is a Gr\"obner basis for I, this implies that $\LT(g) \not \in \LT(I)$.
  By the definition of $\LT(I)$, we conclude that $g \not \in I$.

  To show the uniqueness of the remainder, suppose $g_1$ and $g_2$ are two remainders obtained by two possible runnings of the algorithm.
  Note that $f-g_1$ and $f-g_2$ both lie in $I$, and hence so does $g_1-g_2$.
  To get a contradiction, assume that $g_1 \neq g_2$.
  Then $\LT(g_1 - g_2)$ is not divisible by any $\LT(f_i)$ for $i = 1,\dots,n$ (why?).
  Hence, we have
  \[\LT(g_1-g_2) \not \in \langle \LT(f_1), \dots, \LT(f_n) \rangle.\]
  Again, since $f_1, \dots, f_n$ is a Gr\"obner basis, we have $\LT(g_1-g_2) \not \in \LT(I)$, and hence $g_1 -g_2 \not \in I$.
  This contradicts our previous observation that $g_1-g_2 \in I$.
\end{proof}
Note that \autoref{prop:divgrob} says, in particular, that a Gr\"obner basis of $I$ does actually generate $I$ (this was not a part of the definition!).

\subsection{Elimination theory}
\autoref{prop:divgrob} says that a Gr\"obner basis for $I$ gives an algorithm to test membership in $I$, and in particular, gives an algorithm to test whether $V(I) \neq \emptyset$.
If we choose $\prec$ wisely, it also allows us to find the solutions to the polynomials generating $I$.
We illustrate this in an example first, and then formalize it.

Consider the system of equations
\begin{align*}
  x^2-y-z &= 1,\\
  x-y^2-z &= 1,\\
  x-y-z^2 &= 1.
\end{align*}
To solve such a system, we would employ a technique from linear algebra of eliminating variables one by one until we get an equation in one variable, solve it (by whatever method), back-substitute the solution in the previous equation, solve it for the new variable, back-substitute, and proceed.
Applied to linear equation, this is the method of Gaussian elimination.
It is much less clear how this should be applied this to polynomial equations, let alone that it should always work.

Let take the lexicographical ordering $\prec = \prec_{\rm lex}$ with $z \prec y \prec x$.
The Gr\"obner basis of the ideal
\[ I= \langle  x^2-y-z-1, x-y^2-z-1, x-y-z^2 - 1 \rangle\]
with respect to this ordering is
\begin{align*}
  f_1 &= x-y-z^2 - 1\\
  f_2 &= y^2-y-z^2-z \\
  f_3 &= 2yz^2-z^4-z^2\\
  f_4 &= z^6-4z^4-4z^3-z^2.
\end{align*}
We see that the Gr\"obner basis has done the elimination job for us.
The new system is in the form suited for back-substitution!

This is not an accident.
A Gr\"obner basis with respect to $\prec_{\rm lex}$ always produces an ``upper triangular'' system of generators.
To make this precise, let us take a deeper look at what we mean by eliminating a variable (or a set of variables).
Let $I \subset k[x_1,\dots,x_s]$ be an ideal; think of it as the ideal generated by some sequence of polynomial equations.
Eliminating $x_1$ means finding linear combinations of these equations that do not contain the variable $x_1$.
Define the $\ell$th \emph{elimination ideal} of $I$ by
\[ I_{\ell} = I \cap k[x_{\ell+1}, \dots, x_s].\]
Note that $I_{\ell}$ is an ideal of $k[x_{\ell+1},\dots,x_s]$.
It consists of precisely those linear combinations of the generators of $I$ that do not involve the variables $x_1, \dots, x_{\ell}$.
Therefore, eliminating $x_1, \dots, x_{\ell}$ means finding an explicit set of generators for $I_{\ell}$.
The following proposition says that a G\"obner basis with respect to $\prec_{\rm lex}$ does the trick.
\begin{proposition}
  Let $G$ be a Gr\"obner basis for $I$ with respect to $\prec_{\rm lex}$.
  Then $G \cap k[x_{\ell+1}, \dots, x_s]$ is a Gr\"obner basis for $I_{\ell}$.
\end{proposition}
\begin{proof}
  Let $g \in I_{\ell}$ be non-zero.
  We must show that $\LT(g)$ is divisible by $\LT(f_i)$ for some $f_i$ in $G \cap k[x_{\ell+1},\dots,x_s]$.
  Since $g \in I$, and $G$ is a Gr\"obner basis, we know that $\LT(g)$ is divisible by $\LT(f_i)$ for some $f_i \in G$.
  By the properties of a term ordering, we have
  \[ \LT(f_i) \prec_{\rm lex} \LT(g).\]
  Since $g$ does not involve $x_1, \dots, x_\ell$, and we have a lexicographical ordering with $x_s \prec \dots \prec x_1$, this inequality means that $\LT(f_i)$ does not involve $x_1, \dots, x_{\ell}$.
  Therefore, $f_i$ does not involve $x_1, \dots, x_{\ell}$.
  That is, $f_i$ lies in $G \cap k[x_{\ell+1}, \dots, x_s]$.
  The proof is now complete.  
\end{proof}

\subsection{Buchberger's algorithm}
How do we know whether a given set of generators is a Gr\"obner basis?
If it is not, how do we extend it to one?
The answers to both questions are easy, and involve the notion of an $S$-polynomial.

Fix a term ordering $\prec$.
Let $f$ and $g$ be two non-zero polynomials.
The \emph{S-polynomial} of $f$ and $g$ is the linear combination
\[ S(f,g) = f \cdot \frac{\lcm(\LT(f), \LT(g))}{\LT(f)} - g \cdot \frac{\lcm(\LT(f), \LT(g))}{\LT(g)}. \]
For example, if $f = x^2 + xy$ and $g = xy - 3$, and $\prec = \prec_{\rm grlex}$, then
\begin{align*}
  S(f,g) &= f \cdot \frac{x^2y}{x^2} - g \cdot \frac{x^2y}{xy}\\
         &= f \cdot y - g \cdot x \\
         &=xy^2+3x.
\end{align*}

Let $f_1, \dots, f_n$ be elements of an ideal $I$.
If $f_1, \dots, f_n$ form a Gr\"obner basis, then for all $i$ and $j$, the leading term $\LT(S(f_i,f_j))$ is divisible by $\LT(f_k)$ for some $k$.
The converse is also true, and provides a (finite) set of checks to test whether $f_1, \dots, f_n$ is a Gr\"obner basis.
\begin{proposition}
  A subset $\{f_1, \dots, f_n\} \subset I$ is a Gr\"obner basis if and only if for all $i$ and $j$, there exists a $k$ such that $\LT(S(f_i,f_j))$ is divisible by $\LT(f_k)$.
\end{proposition}
If $\LT(S(f_i,f_j))$ is not divisible by $\LT(f_k)$ for any $k$, then we may add $S(f_i,f_j)$ to the list $\{f_1, \dots, f_n\}$.
Continuing in this way, we may then hope to get a Gr\"obner basis.
This works!

\begin{algorithm}
  \KwData{An ideal $I$ generated by $G = \{f_1, \dots, f_n\}$ and a term order $\prec$}
  \KwResult{A Gr\"obner basis for $I$ with respect to $\prec$}
  \While{
    there exist $f, g \in G$ such that $\LT(S(f,g))$ is not divisible by $\LT(h)$ for any $h \in G$
  }
  {
    Enlarge $G$ by adding $S(f,g)$ to it.
  }
  Return $S$.
  \bigskip

  \caption{Buchberger's algorithm to construct a Gr\"obner basis}
  \label{alg:buch}
\end{algorithm}

\begin{proposition}
  \autoref{alg:buch} stops after finitely many steps and returns a Gr\"obner basis for $I$.
\end{proposition}

\subsection{Computational and numerical costs}


\bibliography{ch2}
\bibliographystyle{amsalpha}

\end{document}